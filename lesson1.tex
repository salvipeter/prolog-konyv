% -*- fill-column: 50 -*-
% (local-set-key (kbd "C-c C-f") 'display-fill-column-indicator-mode)

\chapter{Ismerkedés}

Az első leckében ...

\section{Tények}

Egy Prolog program \emph{tényekből} és
\emph{szabályokból} áll, amikből a számítógép
következtetéseket tud levonni. Ha megadunk egy
tény- és szabályrendszert, utána kérdéseket
tehetünk fel, amiket a gép legjobb tudása szerint
megválaszol.

Példaként vegyük Mohamed próféta családját!
\begin{prolog}
 Abú Tálib --- Abdulla === Ámna
     |                  |
     |                  |
     |               Mohamed === Hadídzsa
     |                        |
     |                        |
    Ali === Fátima ----------------- Zajnab
         |                             |
         |                             |
Huszajn --- Muhszin --- Haszan       Umáma
\end{prolog}
(Ez csak egy nagyon kis részlete a családfának, és
még ezen a részen belül sem tartalmaz minden
kapcsolatot, mert pl. Ali Fátima halála után
feleségül vette Umámát is. Egy teljesebb fát
betettem a dokumentum végére.)

A szülő--gyerek kapcsolatokat az alábbi program
foglalja össze:

\begin{prolog}
szülő(abú_tálib, ali).
szülő(abdulla, mohamed).
szülő(ámna, mohamed).
szülő(mohamed, fátima).
szülő(hadídzsa, fátima).
szülő(mohamed, zajnab).
szülő(hadídzsa, zajnab).
szülő(ali, huszajn).
szülő(fátima, huszajn).
szülő(ali, muhszin).
szülő(fátima, muhszin).
szülő(ali, haszan).
szülő(fátima, haszan).
szülő(zajnab, umáma).
\end{prolog}

Ebben a programban minden sor egy \emph{tény}. Egy
tény dolgok (itt emberek) közti kapcsolatot ír
le. A formája a következő: adunk neki valami nevet
(most ez a \pr{szülő}), utána zárójelek között
vesszővel elválasztva felsoroljuk a kapcsolatban
levő dolgokat, és a végén egy ponttal (\pr{.})
zárjuk.\index{tény}

Talán furcsa lehet, hogy minden kisbetűvel van --
majd később lesz szó arról, hogy mik az elnevezés
pontos szabályai, egyelőre azt jegyezzétek meg,
hogy minden \emph{konkrét} dolog kisbetűvel
írandó, és nem lehet benne szóköz.

\subsection*{Egyszerű kérdések}

Már egy ilyen egyszerű program esetén is lehet
értelmes kérdéseket feltenni. Például
megkérdezhetjük, hogy
\begin{prolog}
?- szülő(mohamed, fátima).
\end{prolog}
... amire a rendszer a \pr{true} (igaz) üzenettel
válaszol, vagy hogy
\begin{prolog}
?- szülő(ali, zajnab).
\end{prolog}
... amire a \pr{false} (hamis) eredményt adja.

Egy kicsit érdekesebb a következő kérdés:
\begin{prolog}
?- szülő(mohamed, Kicsoda).
\end{prolog}
Erre azt a feleletet kapjuk, hogy \pr{Kicsoda = fátima}. Ha rányomtok az eredmény alatt levő \pr{Next} (következő) gombra, akkor még azt is kiírja alá, hogy \pr{Kicsoda = zajnab}.

Mi történt itt? Azáltal, hogy a második helyre egy nagybetűs szót (\pr{Kicsoda}) írtunk, azt mondtuk, hogy ez egy határozatlan, ismeretlen érték. A kérdést tehát magyarul úgy lehetne megfogalmazni: ,,Mohamed kinek a szülője?''

Erre alapból megkeresi az első választ, amit talál (\pr{fátima}), és ha továbbit kérünk tőle, akkor megtalálja \pr{zajnab}-ot is, és látja, hogy nincs több, és így leáll.

Ha kisbetűvel írtuk volna:
\begin{prolog}
?- szülő(mohamed, kicsoda).
\end{prolog}
... akkor a \pr{false} eredményt kaptuk volna, hiszen \pr{kicsoda} itt egy konkrét dolgot jelöl, a kérdés tehát azt jelenti: ,,Igaz-e, hogy Mohamed szülője Kicsodának?''

A kérdés a másik irányban is feltehető:
\begin{prolog}
?- szülő(Ki, ali).
\end{prolog}
... tehát ,,Ki Ali szülője?'', amire a \pr{Ki = abú\_tálib} feleletet kapjuk.

Még tovább is mehetünk, és rákérdezhetünk az összes szülő-gyerek kapcsolatra:
\begin{prolog}
?- szülő(Ki, Kinek).
\end{prolog}
... azaz ,,Ki kinek a szülője?''. Az első válasz az lesz, hogy
\begin{prolog}
Ki = abú_tálib,
Kinek = ali
\end{prolog}
... és a \pr{Next} nyomogatásával a többit is megkaphatjuk. (A \pr{10} és \pr{100} gombok rendre a következő 10 ill. 100 megoldást mutatják meg.)

\subsection*{Összetett kérdések}

Tegyük fel, hogy arra vagyunk kíváncsiak, hogy kik
Mohamed unokái. Hogyan tudjuk ezt megkérdezni? Az
unoka az a gyerek gyereke, tehát tudjuk, hogy van
valaki, aki szülője az unokának, és akinek a
szülője Mohamed. A logikai \emph{és}\/sel
összekötött, együtt teljesítendő feltételeket
vesszővel választjuk el:
\begin{prolog}
?- szülő(mohamed, Valaki), szülő(Valaki, Unoka).
\end{prolog}
Négy megoldást is talál:
\begin{prolog}
Unoka = huszajn,
Valaki = fátima
Unoka = muhszin,
Valaki = fátima
Unoka = haszan,
Valaki = fátima
Unoka = umáma,
Valaki = zajnab
\end{prolog}
(Ezek páronként értendők, tehát Fátimától van Huszajn, Muhszin és Haszan, és Zajnabtól Umáma.)

A kérdés két tagjának sorrendje felcserélhető, tehát ugyanezt az eredményt adja ez is:
\begin{prolog}
?- szülő(Valaki, Unoka), szülő(mohamed, Valaki).
\end{prolog}
(Majd látni fogjuk viszont, hogy a számításigényük nem azonos, érdemesebb az erősebb megszorítással kezdeni.)

Hasonlóan rákérdezhetünk, hogy kik Huszajn nagyszülei:
\begin{prolog}
?- szülő(Valaki, huszajn), szülő(Nagyszülő, Valaki).
\end{prolog}
... amire megkapjuk Abú Tálibot, Mohamedet és Hadídzsát.

Ha arra vagyunk kíváncsiak, hogy Haszan és Huszajn testvérek-e, így fogalmazhatjuk meg:
\begin{prolog}
?- szülő(Valaki, haszan), szülő(Valaki, huszajn).
\end{prolog}
Azt kapjuk, hogy \pr{Valaki = ali}, tehát a válasz igen. Ha \pr{huszajn} helyett \pr{abdulla}-t írunk, akkor \pr{false}-ot kapunk.

\begin{problem}
Válaszoljátok meg az alábbi kérdéseket először magatok, majd utána ellenőrizzétek a számítógépen!
\begin{enumerate}
\item \pr{?- szülő(huszajn, X).}
\item \pr{?- szülő(X, huszajn).}
\item \pr{?- szülő(ámna, X), szülő(X, fátima).}
\item \pr{?- szülő(ámna, X), szülő(X, Y), szülő(Y, haszan).}
\end{enumerate}
\end{problem}
\begin{problem}
Fogalmazzátok meg Prologban!
\begin{enumerate}
\item Ki Ali szülője?
\item Umámának van gyereke?
\item Ki Zajnab nagyszülője?
\end{enumerate}
\end{problem}
\begin{problem}
Készítsétek el a saját családfátokat (nagyszülőkig
és unokatestvérekig)!
\end{problem}

\section{Szabályok}

Eddig csak tényekkel foglalkoztunk, de valójában a
Prolog programok nagy része \emph{szabályokból}
áll. Először is egészítsük ki a programunkat a
szereplőink nemével!\index{szabály}

\begin{prolog}
férfi(abú_tálib).
férfi(abdulla).
férfi(mohamed).
férfi(ali).
férfi(huszajn).
férfi(muhszin).
férfi(haszan).

nő(ámna).
nő(hadídzsa).
nő(fátima).
nő(zajnab).
nő(umáma).
\end{prolog}

Ha most a \pr{szülő} mellett szeretnénk \pr{anya} és \pr{apa} kapcsolatokat is létrehozni, ezt megtehetjük egyenként:
\begin{prolog}
anya(ámna, mohamed).
anya(hadídzsa, fátima).
...
apa(abú_tálib, ali).
apa(abdulla, mohamed).
...
\end{prolog}
... de ez elég sok munka, és érezzük, hogy felesleges, hiszen kikövetkeztethető.

Sokkal egyszerűbb ezt egy-egy szabállyal megoldani:
\begin{prolog}
anya(X, Y) :- szülő(X, Y), nő(X).
apa(X, Y) :- szülő(X, Y), férfi(X).
\end{prolog}
A \pr{:-} jelet itt úgy olvashatjuk ki, hogy ,,akkor, ha'', tehát ,,X anyja Y-nak akkor, ha X szülője Y-nak és X nő''. A baloldalon levő részt a szabály \emph{fej}\/ének, a jobboldalát a szabály \emph{törzs}\/ének nevezzük.
  
Mi történik, amikor feltesszük az alábbi kérdést?
\begin{prolog}
?- anya(ámna, mohamed).
\end{prolog}
A rendszer megtalálja az \pr{anya} szabályt, és \emph{egyesíti} az \pr{X}-et \pr{ámna}-val, az \pr{Y}-t pedig \pr{mohamed}-del. (Az egyesítésről még később szó lesz, itt egyszerűen helyettesítést jelent.) Ezután megnézi, hogy a szabály jobboldalán levő feltétel teljesül-e; ez most \pr{szülő(ámna, mohamed), nő(ámna)}, és ezek a tények szerepelnek a programban, tehát \pr{true}-val tér vissza.

Definiáljuk a ,,nagyszülő'' kapcsolatot!
\begin{prolog}
nagyszülő(X, Z) :- szülő(X, Y), szülő(Y, Z).
\end{prolog}
Ez pontosan követi azt, ahogy megkerestük valakinek a nagyszülőjét.

\subsection*{Problémás esetek}

Hogyan tudnánk megadni a ,,fivér'' kapcsolatot?
\begin{prolog}
fivér(X, Y) :- szülő(Z, X), szülő(Z, Y), férfi(X).
\end{prolog}
Tehát X fivére Y-nak, ha van egy közös szülőjük és X férfi.

Itt érdemes megjegyezni, hogy bár Abú Tálib és Abdulla testvérek voltak, a
\begin{prolog}
?- fivér(abú_tálib, abdulla).
\end{prolog}
kérdésre \pr{false} a válasz, mivel a programnak nincs arról tudomása, hogy lenne közös szülőjük. (A Prolog mindent hamisnak vesz, amit az általa ismert adatokból nem tud kikövetkeztetni, és ez időnként furcsa következményekkel járhat - erről majd később.)

Egy másik problémába ütközünk, ha Mohamed fivéreire vagyunk kíváncsiak:
\begin{prolog}
?- fivér(mohamed, X).
\end{prolog}
Az eredmény, meglepő módon, \pr{X = mohamed}! Ebből látszik, hogy a definíciónk nem volt elég pontos; hozzá kell venni azt is, hogy senki nem fivére önmagának:
\begin{prolog}
fivér(X, Y) :- szülő(Z, X), szülő(Z, Y), férfi(X), X \= Y.
\end{prolog}
Itt a \pr{\textbackslash=} jelentése ,,nem azonos''.

\subsection*{Többféle olvasat}

Készítsük el a
\begin{prolog}
vangyereke(X) :- szülő(X, Y).
\end{prolog}
szabályt.

Ennek az az érdekessége, hogy többféleképpen is ki lehet olvasni:

\begin{enumerate}
\item Ha X szülője Y-nak, akkor X-nek van gyereke.
\item X-nek van gyereke, ha X szülője valakinek.
\end{enumerate}
A két olvasat egyenértékű.

Ha ezt a szabályt beírjátok a programba, egy figyelmeztetés fog megjelenni, hogy az \pr{Y} egy ,,singleton variable'', azaz egy egyszeri változó. Ez csak annyit jelent, hogy az \pr{Y}-t sehol nem használjuk, ezért nem fontos neki nevet adni. Az ilyen névtelen változók helyett az alsóvonás (\pr{\_}) jelet szokás használni, vagy egy ezzel kezdődő nevet (pl. \pr{\_Y}). Ha ilyenre átírjátok, akkor a figyelmeztetés is eltűnik.

\begin{problem}
Fordítsátok le Prologra!
\begin{enumerate}  
\item Akinek van gyereke, az boldog. (\pr{boldog} szabály)
\item Minden X-re, ha X-nek van egy gyereke akinek van egy fivére, akkor X-nek két gyereke van. (\pr{kétgyerekes} szabály)
\end{enumerate}
\end{problem}
\begin{problem}
  Készítsétek el az \pr{unoka} szabályt! Teszteljétek a saját családfátokon!
\end{problem}
\begin{problem}
  Csináljatok egy \pr{nagybácsi} szabályt a \pr{szülő} és \pr{fivér} segítségével! Keressétek meg vele Robin unokahúgait és unokaöccseit!
\end{problem}

\section{Rekurzív szabályok}

Adjunk még egy utolsó szabályt a programunkhoz: az \emph{ős} fogalmát. Valakinek az őseit úgy kapjuk meg, hogy felfelé megyünk a családfában: ős a szülő, a nagyszülő, a dédszülő stb. Ezt elkezdhetjük írni szabályokkal:
\begin{prolog}
ős(X, Z) :- szülő(X, Z).
ős(X, Z) :- szülő(X, Y), szülő(Y, Z).
ős(X, Z) :- szülő(X, Y1), szülő(Y1, Y2), szülő(Y2, Z).
ős(X, Z) :- szülő(X, Y1), szülő(Y1, Y2), szülő(Y2, Y3), szülő(Y3, Z).
...
\end{prolog}
Ez nagyon jól működik, de véges - bármennyi ilyen programsort írok, mindig tudok eggyel feljebb menni a családfában, és azt már nem kezeli.

Egy kis trükkel meg tudjuk ezt oldani: azt mondjuk, hogy ha X gyereke őse Z-nek, akkor X is őse Z-nek:
\begin{prolog}
ős(X, Z) :- szülő(X, Y), ős(Y, Z).
\end{prolog}
Ez így magában még azonban nem elég, mert így ahhoz, hogy valaki ős legyen, mindig valaki másnak is ősnek kéne lennie, valahol ennek meg kéne állnia. Elég hozzávenni a legegyszerűbb esetet, amikor a szülő az ős:
\begin{prolog}
ős(X, Z) :- szülő(X, Z).
ős(X, Z) :- szülő(X, Y), ős(Y, Z).
\end{prolog}
Ez a kettő együtt már működik. X őse Z-nek, ha vagy (i) X szülője Z-nek, vagy (ii) X szülője Y-nak és Y őse Z-nek. Próbáljátok ki, mit ad az
\begin{prolog}
?- ős(X, huszajn).
\end{prolog}
kérdés!

\begin{problem}
Tegyük fel, hogy az \pr{ős} definícióját megváltoztatjuk!
\begin{prolog}
ős(X, Z) :- szülő(X, Z).
ős(X, Z) :- szülő(Y, Z), ős(X, Y).
\end{prolog}
Jó ez így? Miért?
\end{problem}

\section*{A teljes program}

Itt van minden tény és szabály, amiről szó volt. A
programban a \pr{\%} jel megjegyzések hozzáadására
használható, a rendszer szempontjából a \pr{\%}-tól
jobbra levő szöveg érdektelen, mintha ott se
lenne.

\begin{prolog}
% szülő(X, Y) -> X az Y szülője
szülő(abú_tálib, ali).
szülő(abdulla, mohamed).
szülő(ámna, mohamed).
szülő(mohamed, fátima).
szülő(hadídzsa, fátima).
szülő(mohamed, zajnab).
szülő(hadídzsa, zajnab).
szülő(ali, huszajn).
szülő(fátima, huszajn).
szülő(ali, muhszin).
szülő(fátima, muhszin).
szülő(ali, haszan).
szülő(fátima, haszan).
szülő(zajnab, umáma).

férfi(abú_tálib).
férfi(abdulla).
férfi(mohamed).
férfi(ali).
férfi(huszajn).
férfi(muhszin).
férfi(haszan).

nő(ámna).
nő(hadídzsa).
nő(fátima).
nő(zajnab).
nő(umáma).

anya(X, Y) :- szülő(X, Y), nő(X).     % X az Y anyja
apa(X, Y) :- szülő(X, Y), férfi(X).   % X az Y apja

nagyszülő(X, Z) :- szülő(X, Y), szülő(Y, Z).

fivér(X, Y) :- szülő(Z, X), szülő(Z, Y), férfi(X), X \= Y.

vangyereke(X) :- szülő(X, _).

ős(X, Z) :- szülő(X, Z).
ős(X, Z) :- szülő(X, Y), ős(Y, Z).
\end{prolog}

\begin{infobox}{.75}{foo és bar}
  Érdekes infók Sed at magna at sem tempus
  mattis. Duis et quam sed metus vestibulum
  vulputate eget id eros. Curabitur vel consequat
  purus. Etiam finibus, augue et hendrerit
  imperdiet, mi mauris aliquam tortor, ultrices
  interdum sem augue sed diam.
\end{infobox}
\section{Projekt: bla}
Blabla
