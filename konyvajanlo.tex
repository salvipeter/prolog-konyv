\chapter{Könyvajánló}
Magyar nyelven nem sok minden jelent meg a Prologról, bár fontos
megemlíteni a Szeredi Péter által fejlesztett \name{Mprolog}
(\emph{Modular Prolog}) rendszerről szóló
könyvet:
\begin{itemize}
\item[] Zs.~Farkas, I.~Futó, T.~Langer, P.~Szeredi, \emph{Mprolog
programozási nyelv}, Műszaki Könyvkiadó, 1989.
\end{itemize}
Angolul már sokkal nagyobb a választék. A programozási készség
elsajátításához a legfontosabb a gyakorlás; rengeteg érdekes feladat
található (megoldásokkal!) az alábbiakban:
\begin{itemize}
\item[] H.~Coelho, J.~C.~Cotta, \emph{Prolog by Example---How to Learn, Teach and Use It}, Springer, 1988.
\item[] B.~Demoen, Ph-L.~Nguyen, T.~Schrijvers, R.~Tron\c con, \emph{The First 10 Prolog Programming Contests}, Belgium, 2005.
\end{itemize}
Hatékony programok írásához sok hasznos és praktikus tanács-csal szolgál ez a könyv:
\begin{itemize}
\item[] R.~A.~O'Keefe, \emph{The Craft of Prolog}, MIT Press, 1990.
\end{itemize}
Nem maradhat ki a listából Ivan Bratko klasszikusa sem,
ami bár elsősorban mesterséges intelligenciai algoritmusokról szól, de emellett
tartalmaz egy nagyszerű bevezetőt is a Prolog programozásba:
\begin{itemize}
\item[] I.~Bratko: \emph{Prolog Programming for Artificial Intelligence}, 4th Ed., Pearson, 2011.
\end{itemize}
(A leckék nagy vonalakban e könyv első részének felépítését követték.)

\bigskip
\begin{center}
\rule{0.5\textwidth}{.5pt}
\end{center}
\bigskip

Bár a Prolog nyelvhez nem kapcsolódik közvetlenül, de egy rendkívül
olvasmányos és változatos áttekintést ad a számítástechnikáról a
Turing Omnibus:
\begin{itemize}
\item[] A.~K.~Dewdney, \emph{The (New) Turing Omnibus}, Freeman / Holt, 1993.
\end{itemize}
Végül pedig azoknak, akik komolyabban érdeklődnek a téma iránt, egy jó
kiindulási pont a \emph{SICP} avagy a ,,varázslós könyv'':
\begin{itemize}
\item[] H.~Abelson, G.~J.~Sussman, J.~Sussman, \emph{Structure and Interpretation of Computer Programs}. MIT Press, 1996.
\end{itemize}
