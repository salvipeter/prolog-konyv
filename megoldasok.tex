\chapter{A feladatok megoldásai}
\subsubsection*{1.~feladat}
\begin{query}
?- szülő(huszajn, X).
false

?- szülő(X, huszajn).
X = ali ;
X = fátima

?- szülő(ámna, X), szülő(X, fátima).
X = mohamed

?- szülő(ámna, X), szülő(X, Y), szülő(Y, haszan).
X = mohamed,
Y = fátima
\end{query}
\subsubsection*{2.~feladat}
\begin{query}
?- szülő(X, ali).
?- szülő(umáma, X).
?- szülő(X, zajnab), szülő(Nagyszülő, X).
\end{query}
\subsubsection*{4.~feladat}
\begin{program}
boldog(X) :- vangyereke(X).
kétgyerekes(X) :- szülő(X, Y), fivér(Y, _).
\end{program}
\subsubsection*{5.~feladat}
\begin{program}
unoka(X, Y) :- nagyszülő(Y, X).
\end{program}
\subsubsection*{6.~feladat}
\begin{program}
nővér(X, Y) :-
    nő(X),
    szülő(Z, X), szülő(Z, Y),
    X \= Y.
nagynéni(X, Y) :- nővér(X, Z), szülő(Z, Y).
\end{program}
\begin{query}
?- nagynéni(zajnab, X), férfi(X).
\end{query}
\subsubsection*{7.~feladat}
Jó a definíció; X őse Z-nek, ha (i) X szülője Z-nek, vagy (ii) X őse Z szülőjének.
\subsubsection*{8.~feladat}
