\chapter{A feladatok megoldásai}
\subsubsection*{1.~feladat}
\begin{query}
?- szülő(huszajn, X).
false

?- szülő(X, huszajn).
X = ali ;
X = fátima

?- szülő(ámna, X), szülő(X, fátima).
X = mohamed

?- szülő(ámna, X), szülő(X, Y), szülő(Y, haszan).
X = mohamed,
Y = fátima
\end{query}
\subsubsection*{2.~feladat}
\begin{query}
?- szülő(X, ali).
?- szülő(umáma, X).
?- szülő(X, zajnab), szülő(Nagyszülő, X).
\end{query}
\subsubsection*{4.~feladat}
\begin{program}
boldog(X) :- vangyereke(X).
kétgyerekes(X) :- szülő(X, Y), fivér(Y, _).
\end{program}
\subsubsection*{5.~feladat}
\begin{program}
unoka(X, Y) :- nagyszülő(Y, X).
\end{program}
\subsubsection*{6.~feladat}
\begin{program}
nővér(X, Y) :-
    nő(X),
    szülő(Z, X), szülő(Z, Y),
    X \= Y.
nagynéni(X, Y) :- nővér(X, Z), szülő(Z, Y).
\end{program}
\begin{query}
?- nagynéni(zajnab, X), férfi(X).
\end{query}
\subsubsection*{7.~feladat}
Jó a definíció; X őse Z-nek, ha (i) X szülője Z-nek, vagy (ii) X őse Z szülőjének.
\subsubsection*{8.~feladat}
Változó; atom; atom; változó; atom; struktúra; szám; (hibás); struktúra; (hibás).
\subsubsection*{9.~feladat}
Erre a feladatra nincs \emph{egyetlenegy} jó megoldás; különböző alkalmazásokhoz
más és más leírási módok lehetnek kényelmesebbek.
\begin{itemize}
\item Egy tengelyekkel párhuzamos téglalapot le lehet (pl.) írni a bal
  felső és jobb alsó sarkával, tehát \pr{téglalap(BalFelső, JobbAlsó)},
  ahol \pr{BalFelső} és \pr{JobbAlsó} pontok: \pr{pont(X, Y)}.  Ha a
  téglalap a tengelyekkel nem párhuzamos, akkor a legegyszerűbb talán
  mind a négy csúcspontot felsorolni (bár ez redundáns).
\item Egy négyzetet mindig le lehet írni a bal felső és jobb alsó
  sarokkal, de lehet pl.~a középponttal és egy csúcsponttal is;
  tengelyekkel párhuzamos esetben elég a középpont és a csúcsok ettől
  való távolsága is.
\item Egy kört is reprezentálhat a befoglaló négyzete, de megadható a
  középpontja és a sugara által is: \pr{kör(Középpont, Sugár)},
  pl.~\pr{kör(pont(1,2),3)}.
\end{itemize}
\subsubsection*{10.~feladat}
