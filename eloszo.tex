\chapter{Előszó}
Ez a könyv elsősorban azoknak szól, akik még egyáltalán nem tudnak
programozni, de szeretnék megismerni a számítógépes problémamegoldás
alapjait. A ,,programozás'' ma már egy rendkívül szerteágazó
gyűjtőfogalommá vált, melynek egy-egy területe magában egy-egy szakmát
képvisel. Itt nem lesz szó webfejlesztésről, látványos játékokról vagy
neurális hálókról: kizárólag a problémamegoldásra fogunk koncentrálni,
ami viszont -- az én véleményem szerint -- a programozás
legizgalmasabb válfaja.

Ehhez a Prologot fogjuk használni. Ez az 50 éves múltra visszatekintő
programozási nyelv sokáig állt a mesterséges intelligencia kutatás
középpontjában, és bár az iparban nem hódított teret, az informatika
oktatásában továbbra is jelentős szerepe van. Ez nem véletlen: a
Prolog különösen alkalmas tanításra, ugyanis egészen minimális
ismeretekkel is már érdekes feladatok megoldását teszi lehetővé. Ezt
bizonyítják az egyes leckéket lezáró \emph{projektek} is, amelyekben
különböző, kicsit nagyobb lélegzetvételű problémákat vizsgálunk majd
meg.

\section*{Hogyan használjuk a könyvet?}
A leckékben szereplő példákat érdemes a számítógépen kipróbálni, és
kísérletezgetni velük, átírni bennük ezt--azt, és megvizsgálni, hogy mi
történik. Az efféle játék során mélyül el igazán a tudás; és ezt
segítik a feladatok is, melyeknek megoldása a könyv végén található.

Mindehhez természetesen szükség van egy Prolog fordítóra. Ezekből sok
létezik, és mindegyik meglehetősen különböző. A legtöbb azonban
kompatibilis az 1995-ös ISO szabvánnyal, és ez a könyv csak ezt
feltételezi, illetve még annyit, hogy tudja kezelni a magyar ábécé
betűit. Az ingyenesen letölthető, nyílt forráskódú rendszerek közül
ilyen az \name{SWI-Prolog}. Más implementációk, mint pl. az~\name{XSB}
vagy a \name{GNU Prolog}, jelenleg nem fogadják el az ékezetes
karaktereket, ezért ezek használata esetén a programokat ékezetek
nélkül kell begépelni.

A problémamegoldás Prologban mindig két részből áll:
\begin{enumerate}
\item Egy -- általában {\tt .pl} vagy {\tt .P} kiterjesztésű --
  fájlban megadunk egy szabályrendszert (a program
  \emph{forráskódját}). Ennek megírásához tetszőleges
  szövegszerkesztőt használhatunk.\index{forráskód}
\item Ezzel kapcsolatban a programnak felteszünk kérdéseket.
\end{enumerate}
A legtöbb Prolog rendszer indításkor mindössze egy kérdőjel kiírásával
fogad minket; ezzel jelzi, hogy várja a kérdést:
\begin{query}
?-
\end{query}  
Az első feladatunk, hogy betöltsük az általunk írt programot. Ezt
általában a \pr{consult} paranccsal tehetjük meg, például:\index{\pr{consult}}
\begin{query}
?- consult('/home/salvi/szabályok.pl').
\end{query}
\dots ahol az aposztrófok között levő {\tt /home/salvi/szabályok.pl} a
forráskód teljes elérési útja. Ne felejtsük le a parancs végéről a
pontot, majd nyomjuk meg az újsor billentyűt.

Ezután már jöhetnek a kérdések. A leckékben már csak ezek lesznek
feltüntetve, de nem szabad megfeledkezni a szabályok betöltéséről sem.
Amennyiben egy kérdésre több válasz is van, a következő választ
általában a pontosvessző ({\tt ;}) lenyomásával kaphatjuk meg, a
következő kérdés feltevéséhez pedig ilyenkor egy újsor billentyűt kell
nyomni.

Bizonyos rendszerekben mindez kényelmesebben is megtehető, így pl.~az
\name{SWI-Prolog} internetes \name{SWISH} verziójában,\footnote[2]{\tt
https://swish.swi-prolog.org/} ahol a szabályok mindig automatikusan
újraolvasódnak.

\section*{Források}
Ezt a könyvet eredetileg azzal a céllal kezdtem el, hogy az egyik
kedvenc programozásról szóló könyvemet~[1] elérhetőbbé és köny-nyebben
emészthetővé tegyem a magyar laikus közönség számára. Ez aztán végül
egy Prolog tankönyvvé vált, de a leckék -- helyenként jelentősen
átdolgozva -- az eredetit követik. Egészen pontosan az 1.~lecke az
1.1--3 és 1.8, a 2.~lecke a 2.1--6, a 3.~lecke a 3.1--2, a 4.~lecke a
3.3--4, az 5.~lecke az 5.1--4 és 9.1, a 6.~lecke pedig a 6.1--6 és 8.5
fejezeteknek felel meg.

Egy másik fontos forrás egy klasszikus Prolog tankönyv~[2]; ebből
származik az 5.~lecke összefésüléses példája (11.~fejezet), a
6.~leckében a különbség-listák tárgyalása (15.1 fejezet), és a
lecke végén található projekt (20.2 és 21.3 fejezetek).

A 4.~leckében levő kígyó-kocka ötletét egy Haskell megoldás~[3] adta;
a 7.~lecke alapját képező forráskód pedig az \name{SWI-Prolog}
szoftvercsomag része.

\begin{itemize}[leftmargin=2cm,itemindent=-1cm,labelsep=1cm-2em]
\item[{[1]}] I.~Bratko: \emph{Prolog Programming for Artificial
Intelligence}, 4th Ed., Pearson, 2011.
\item[{[2]}] L.~Sterling, E.~Shapiro, \emph{The Art of Prolog}, 2nd
  Ed., MIT Press, 1994.
\item[{[3]}] M.~P.~Jones: \emph{Solving the Snake Cube Puzzle in Haskell}.
  Journal of Functional Programming 23(2), pp.~145--160, 2013.
\end{itemize}

\section*{Köszönetnyilvánítás}
Köszönettel tartozom mindenekelőtt Varga Csillának, aki lelkesen
végigcsinálta a könyvnek egy korai változatát, és kérdéseivel
rávilágított arra, hogy mit kell még pontosítanom vagy jobban
elmagyaráznom; és természetesen a feleségemnek, Iida Norikónak, akinek
a támogatása nélkül ez a könyv nem készülhetett volna el.

\begin{flushright}
  Salvi Péter\\
  Budapest, 2022.
\end{flushright}
