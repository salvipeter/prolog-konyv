\chapter{Előszó}
Ez a könyv elsősorban azoknak szól, akik még egyáltalán nem tudnak
programozni, de szeretnék megismerni a számítógépes problémamegoldás
alapjait. A ,,programozás'' ma már egy rendkívül szerteágazó
gyűjtőfogalommá vált, melynek egy-egy területe magában egy-egy szakmát
képvisel. Itt nem lesz szó webfejlesztésről, látványos játékokról vagy
neurális hálókról: kizárólag a problémamegoldásra fogunk koncentrálni,
ami viszont -- az én véleményem szerint -- a programozás
legizgalmasabb válfaja.

Ehhez a Prologot fogjuk használni. Ez az 50 éves múltra visszatekintő
programozási nyelv sokáig állt a mesterséges intelligencia kutatás
középpontjában, és bár az iparban nem hódított teret, az informatika
oktatásában továbbra is jelentős szerepe van. Ez nem véletlen: a
Prolog különösen alkalmas tanításra, ugyanis egészen minimális
ismeretekkel is már érdekes feladatok megoldását teszi lehetővé. Ezt
bizonyítják az egyes leckéket lezáró \emph{projektek} is, amelyekben
különböző, kicsit nagyobb lélegzetvételű problémákat vizsgálunk majd
meg.

\section*{Hogyan használjuk a könyvet?}
A leckékben szereplő példákat érdemes a számítógépen kipróbálni, és
kísérletezgetni velük, átírni ezt--azt, és megvizsgálni, hogy mi
történik. Az efféle játék során mélyül el igazán a tudás; és ezt
segítik a feladatok is, melyeknek megoldása a könyv végén található.

Mindehhez természetesen szükség van egy Prolog fordítóra. Ezekből sok
létezik, és mindegyik meglehetősen különböző. A legtöbb azonban
kompatibilis az 1995-ös ISO szabvánnyal, és ez a könyv csak ezt
feltételezi, illetve még annyit, hogy tudja kezelni a magyar ábécé
betűit. Az ingyenesen letölthető, nyílt forráskódú rendszerek közül
ilyen pl.~az \name{SWI-Prolog} és az~\name{XSB}, illetve a
\name{GNU Prolog} programnak is létezik Unicode-kom\-pa\-ti\-bi\-lis
változata, ami elfogadja az ékezetes karaktereket.

A problémamegoldás Prologban mindig két részből áll:
\begin{enumerate}
\item Egy -- általában {\tt .pl} kiterjesztésű -- fájlban megadunk egy
  szabályrendszert (a program \emph{forráskódját}). Ennek megírásához
  tetszőleges szövegszerkesztőt használhatunk.\index{forráskód}
\item Ezzel kapcsolatban a programnak felteszünk kérdéseket.
\end{enumerate}
A legtöbb Prolog rendszer indításkor mindössze egy kérdőjel kiírásával
fogad minket; ezzel jelzi, hogy várja a kérdést:
\begin{prolog}
?-
\end{prolog}  
Az első feladatunk, hogy betöltsük az általunk írt programot. Ezt
általában a \pr{consult} paranccsal tehetjük meg, például:\index{{\tt consult}}
\vspace{-1em} % hack
\begin{prolog}
?- consult('/home/salvi/szabályok.pl').
\end{prolog}
\dots ahol {\tt /home/salvi/szabályok.pl} a forráskód teljes elérési
útja. Ne felejtsük le a parancs végéről a pontot, majd nyomjuk meg az
újsor billentyűt.

Ezután már jöhetnek a kérdések. A leckékben már csak ezek lesznek
feltüntetve, de nem szabad megfeledkezni a szabályok betöltéséről sem.
Amennyiben egy kérdésre több válasz is van, a következő választ
általában a pontosvessző ({\tt ;}) lenyomásával kaphatjuk meg, a
következő kérdés feltevéséhez pedig ilyenkor egy újsor billentyűt kell
nyomni.

Bizonyos rendszerekben mindez kényelmesebben is megtehető, így pl.~az
\name{SWI-Prolog} internetes \name{SWISH} verziójában,\footnote{\tt
https://swish.swi-prolog.org/} ahol a szabályok mindig automatikusan
újraolvasódnak.

\section*{Köszönetnyilvánítás}
Köszönettel tartozom mindenekelőtt Varga Csillának, aki lelkesen
végigcsinálta a könyvnek egy korai változatát, és kérdéseivel
rávilágított arra, hogy mit kell még pontosítanom vagy jobban
elmagyaráznom; és természetesen a feleségemnek, Iida Norikónak, akinek
a támogatása nélkül ez a könyv nem készülhetett volna el.

\begin{flushright}
  Salvi Péter\\
  Budapest, 2022.
\end{flushright}

\clearpage
\thispagestyle{empty}
